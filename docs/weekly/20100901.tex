\documentclass[11pt,letterpaper,oneside,reqno]{article}
\usepackage[top=2.5cm,bottom=2.5cm,left=2.5cm,right=2.5cm]{geometry}
\usepackage[T1]{fontenc}
\usepackage[latin1]{inputenc}
\usepackage{fancyhdr}
\pagestyle{fancy}
\usepackage{subfigure}
\usepackage{amsmath}
\usepackage{float,graphicx}
\usepackage{natbib}

\usepackage{fourier}

%\usepackage{stix}
%\usepackage{ccfonts,eulervm}
%\usepackage[T1]{fontenc}
\renewcommand{\rmdefault}{put}
%\geometry{letterpaper}
\usepackage[parfill]{parskip}

%\usepackage[intlimits]{amsmath}

\usepackage{setspace}
\usepackage{amsthm}
\usepackage{appendix}

%\usepackage{fancyvrb}
%\usepackage{supertabular}

\usepackage[font={bf,footnotesize},textfont=md,margin=30pt,aboveskip=0pt,belowskip=0pt]{caption}

\usepackage{lastpage}

\DeclareCaptionLabelFormat{appcaption}{#1 \oldthesection-#2}
\setlength{\headheight}{15pt}

\renewcommand{\topfraction}{0.9}    % max fraction of floats at top
    \renewcommand{\bottomfraction}{0.8} % max fraction of floats at bottom
    %   Parameters for TEXT pages (not float pages):
    \setcounter{topnumber}{2}
    \setcounter{bottomnumber}{2}
    \setcounter{totalnumber}{4}     % 2 may work better
    \setcounter{dbltopnumber}{2}    % for 2-column pages
    \renewcommand{\dbltopfraction}{0.9} % fit big float above 2-col. text
    \renewcommand{\textfraction}{0.07}  % allow minimal text w. figs
    %   Parameters for FLOAT pages (not text pages):
    \renewcommand{\floatpagefraction}{0.7}  % require fuller float pages
    % N.B.: floatpagefraction MUST be less than topfraction !!
    \renewcommand{\dblfloatpagefraction}{0.7}   % require fuller float pages

\newenvironment{myenumerate}{
\begin{enumerate}
  \setlength{\itemsep}{1pt}
  \setlength{\parskip}{0pt}
  \setlength{\parsep}{0pt}}{\end{enumerate}
}
\newenvironment{myitemize}{
\begin{itemize}
  \setlength{\itemsep}{1pt}
  \setlength{\parskip}{0pt}
  \setlength{\parsep}{0pt}}{\end{itemize}
}
\usepackage[uline]{hhtensor}
\usepackage{ctable}
%\usepackage{fancyhdr}
\usepackage{listings}

%\usepackage[style=asceish,sorting=nyt,natbib=true]{biblatex}
%\bibliography{capstone_figparts}

\usepackage[pdftex,bookmarks,colorlinks,breaklinks,pdfpagelabels]{hyperref}

\hypersetup{linkcolor=black,citecolor=black,filecolor=black,urlcolor=black}
%\bibliographystyle{ascelike}
%\bibliographystyle{apalike}
%\bibpunct{[}{]}{;}{a}{}{,}
\newcommand{\p}{\partial}
\newcommand{\del}{\nabla}
\renewcommand{\deg}{^\circ}
\newcommand{\comp}{\overline}
\newcommand{\dd}{\; \mathrm{d}}

%
%%%%%%%%%%%%%% Title and TOC %%%%%%%%%%%%%%%%%%%
%

\begin{document}
\centerline{\Large \bf Wave Modeling, Monitoring, and Analysis} 
\centerline{\Large \bf for PG\&E WaveConnect Pilot Project}
\centerline{Colin Sheppard; Nir Berezovsky; Charlie Sharpsteen;}
\centerline{Jim Zoellick; Charles Chamberlin}
\centerline{Schatz Energy Research Center}
\bigskip
\bigskip

\noindent {\bf Subject:} Progress Report

\noindent {\bf To:} Brendan Dooher, PG\&E

\noindent {\bf Submitted by:} Colin Sheppard

\noindent {\bf Date:} September 1, 2010 

\noindent {\bf Attachments:} Current draft of overall project report.

\textbf{Activity to Date}\\
Since the last submitted progress report (August 18, 2010), the team from the Schatz Energy Research Center (SERC) has engaged in the following activities in pursuit of the project goals:
\begin{itemize}
\item Succussfully debugged the CMS-Unix source code to execute model runs on default data sets.
\item Construction of the grid for the Humboldt domain. 
\item Engaged with Army Corps of Engineers about obtaining documentation in input formats.
\item Exteneded the NDBC buoy data downloading script to ingest spectral data.
\item Configured automatic compilation and publishing of code documentation as it is written.
\item Acquired and began installation/configuration of hardware to be used as the primary modeling platform going forward.
\end{itemize}

\textbf{Upcoming Activity}\\
We intend to accomplish the following in the coming weeks (along with expected time horizons until completion):
\begin{itemize}
\item Finalize configuration of CMS and determination of input formats (\texttildelow 1 week).
\item Use data retrieval scripts to download all available data relevant to our modeling effort (\texttildelow 1 weeks).
\item Complete configuration of computer hardware for use in model execution (\texttildelow 1 weeks).
\item Organize travel arrangements to Ocean Renewable Energy Conference, September 28-30 in Portland (\texttildelow 2 weeks).
\item Develop model control scripts in Matlab (\texttildelow 3 weeks).
\item Conduct runs for preliminary resource estimation for the Humboldt domain (\texttildelow 4 weeks).
\end{itemize}

\end{document}
